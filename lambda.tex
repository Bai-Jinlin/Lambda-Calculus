\documentclass[cs4size,a4paper,adobefonts]{ctexart}
\usepackage{amsmath,amsthm,amssymb}
\usepackage[colorlinks=true,allcolors=black]{hyperref}
\usepackage{indentfirst}
\usepackage[a4paper,left=2.5cm,right=2.5cm,bottom=2.5cm,top=2.5cm]{geometry}
\usepackage{graphicx}
\usepackage{subcaption}
\usepackage{url}
\usepackage{fontspec}
\setmainfont{Palatino}
\setmonofont[Scale=MatchLowercase]{Monaco}
\pagestyle{plain}
\punctstyle{kaiming}
\usepackage{unicode-math}
\setmathfont{Asana Math}
\graphicspath{{pic/}}

\begin{document}
\title{{\bfseries 让我们谈谈 $\lambda$}}
\author{\href{mailto:txyyss@gmail.com}{王盛颐}}
\date{}
\maketitle

\section{一点历史}

函数是数学中一个非常基本的概念,出现在几乎所有的数学分支里。为了研究函数的一般性质,美国数学家 Alonzo Church 在 1936 年发明了 $lambda$ 演算这样一个形式系统,并以它为工具解决了 David Hilbert 在 1928 年提出的判定性问题 (Hilbert's Entscheidungsproblem),这比图灵用他的图灵机解决同一问题还早几个月。后来证明 $lambda$ 演算和图灵机是等价的。现在谈到可计算理论的时候,通常还是拿图灵机来作模型,不怎么提 $lambda$ 演算。从这段历史可以知道,$lambda$ 演算被发明出来的时候,没有计算机,更没有程序设计语言。

但好东西到底是好东西,又过了差不多 30 年(谢天谢地那时候已经有计算机和程序语言了),英国计算机科学家 Peter Landin 发现可以通过把复杂的程序语言转化成简单的 $lambda$ 演算,来理解程序语言的行为。这个洞见加上众所周知的 John McCarthy 的 Lisp 语言,让 $lambda$ 演算广为传播。现在不论是各种实际的程序设计语言还是理论上的研究工作,$lambda$ 演算都是一个绕不过去的基本工具,而图灵机也就是在证明可计算性之类的时候用到了。这让我觉得历史到底是公平的,没道理这么漂亮的理论会输给丑陋的图灵机。

$lambda$ 演算之所以这么重要,用 Benjamin C. Pierce 的话说在于它具有某种“二象性”:它既可以被看作一种简单的程序设计语言,用于描述计算过程,也可以被看作一个数学对象,用于推导证明一些命题。所以在这篇介绍 $lambda$ 演算的文章里,我也打算从两个方面来讲,先讲它作为数学理论这方面的内容。

\end{document}
