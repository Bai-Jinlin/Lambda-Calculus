\documentclass[a4paper,adobefonts]{ctexart}
\usepackage[a4paper,margin=2.5cm]{geometry}
\usepackage[colorlinks=true,allcolors=black]{hyperref}
\usepackage[svgnames]{xcolor}
\usepackage{amsmath,amsthm,amssymb}
\usepackage{fancyvrb,fancybox,calc}
\usepackage{fontspec}
\usepackage{graphicx}
\usepackage{indentfirst}
\usepackage{unicode-math}
\usepackage{url}

\setmainfont{Minion Pro}
\setmonofont{Source Code Pro}
\pagestyle{plain}
\punctstyle{kaiming}
\setmathfont{Asana Math}
\graphicspath{{pic/}}

\newcommand{\downloadurl}{\url{http://www.twitter.com}}

\newenvironment{verbcode}{\VerbatimEnvironment%
  \noindent
  \begin{Sbox}
    \begin{minipage}{.96\textwidth}
      \begin{Verbatim}
}{%
      \end{Verbatim}
    \end{minipage}
  \end{Sbox}
  \begin{center}\colorbox{LightGray}{\TheSbox}\end{center}
}

\newenvironment{tightenum}{
  \begin{enumerate}
    \setlength{\itemindent}{2\ccwd}
    \setlength{\itemsep}{0cm}
    \setlength{\parskip}{0cm}
}{
  \end{enumerate}
}

\newtheorem{theorem}{定理}
\theoremstyle{definition}
\newtheorem{definition}{定义}
\newtheorem{example}{示例}
\newtheorem{notation}{符号约定}

\begin{document}
\title{{\bfseries 让我们谈谈 $\lambda$ 演算}}
\author{王盛颐\\\href{mailto:txyyss@gmail.com}{txyyss@gmail.com}}
\date{}
\maketitle

\section*{缘起}

``写篇介绍 $\lambda$ 演算的文章''这一条目在我的待办事项列表里已经躺了
很久了。我本科看``计算机程序的构造和解释''(SICP)一书时,接触到了
$\lambda$ 表达式这个概念。当时我觉得它就是一种表示匿名函数的记法。有了
这个记法,把函数作为值来传递、返回以及组合都方便了很多,别的也没多想。

SICP 里面用的 Scheme 语言让我见识到了世界上还有一类叫 ``函数式'' 的程
序设计语言。搜索相关资料,都说 $\lambda$ 演算是函数式语言的基础。好奇
心害死猫啊,我自然是要看看 $\lambda$ 演算是怎么一回事的。一番了解之后,
我觉得如此形式简单而内涵丰富的东西,就是美的体现啊。尤其当我看到
Church 数的时候,那种``啊哈''的惊艳感真是永生难忘。

后来我学了越来越多的关于计算机,特别是程序设计语言的理论,发现
$\lambda$ 演算从可计算性理论到形式语义到类型理论,无处不在,自己目前所
知不过皮毛罢了。刚好我最近因为学习 Haskell 语言,实现了一个简单的无类
型纯 $\lambda$ 演算解释器。于是我决定写篇初步介绍无类型纯 $\lambda$ 演
算的文章,有个解释器的好处是能一边介绍一边通过解释器演示规约求值的过程,
让相对抽象的 $\lambda$ 演算更加直观和易于理解。

\section{历览前贤}

一切要从上个世纪初,也就是 1900 年左右开始说起了。那时候数学界的气象是
这样的:希尔伯特刚刚提出他的 23 个问题,其中第 2 个问题问的是公理系统
的相容性;1901 年提出的,动摇了集合论基础的罗素悖论还是个很流行的话题;
至于哥德尔不完备性定理呢,大家都还不知道。所以在这之后的几十年很多数学
家和逻辑学家都在致力于给整个数学建立一个一致的公理基础。比如罗素和怀特
海德写了``数学原理''(Principia Mathematica);比如约翰·冯·诺伊曼
\footnote{对,你没有看错,他就是那个定下了计算机基本结构``冯·诺伊曼结
  构''的那个冯·诺伊曼。}写了关于公理化集合论的博士论文;再比如阿隆佐·
邱奇(Alonzo Church)提出了 $\lambda$ 演算。

邱奇是个美国逻辑学家,生于 1903 年。他在 1928 年开始构造的一个形式系统
中包含了纯 $\lambda$ 演算。他发明这一形式系统的初衷是为了给逻辑学提供一
个基础,能代替罗素的类型理论和恩斯特·策梅洛(Ernst Zermelo)
\footnote{他就是 ZFC 公理化集合论的那个``Z''。}的集合理论。这个系统
1932 年发表后不久就被发现有矛盾,于是一年后邱奇修正了一番重新发表。当
时的人对不完备性定理威力有多大还缺乏清醒广泛的认识。所以他还希望那时发
现不久的哥德尔关于``数学原理''一书的不完备性定理不会扩展到他的系统上。

愿望是良好的,结果是残酷的。到 1935 年,邱奇的两个学生,Stephen Kleene
和 Barkley Rosser 发现邱奇的逻辑系统是不一致的。不过柳暗花明又一村,他
们发现系统包含的纯 $\lambda$ 演算则具有一系列良好的性质,再后来更是证
明用 $\lambda$ 演算可以等价的定义出可计算函数,邱奇觉得能有效计算的函
数等价于 $\lambda$ 可定义性,这就是著名的``邱奇--图灵论题''的一部分了。
更一进步的,邱奇用 $\lambda$ 演算证明了一阶逻辑不存在递归判定过程,这
是对希尔伯特提出的判定性问题(Entscheidungsproblem)的第一个否定性答案,
这比用图灵证明停机问题不可判定还要早几个月。

不过二十世纪三十年代在 $\lambda$ 演算方面的成果差不多也就这些了,再接
下来的 20 年都没有太多研究和进展。直到六十年代,那时有了计算机,有了程
序设计语言,有了计算机科学家。在 1965 年,英国计算机科学家 Peter
Landin 发现可以通过把复杂的程序语言转化成简单的$\lambda$ 演算,来理解
程序语言的行为。这个洞见可以让我们把 $\lambda$演算本身看成一种程序设计
语言。而众所周知的 John McCarthy 的 Lisp 语言,更是让 $\lambda$ 演算广
为传播。现在不论是各种实际的程序设计语言还是理论上的研究工作,
$\lambda$ 演算都是一个绕不过去的基本工具了。

$\lambda$ 演算之所以这么重要,用 Benjamin C. Pierce 的话说在于它具有某
种``二象性'':它既可以被看作一种简单的程序设计语言,用于描述计算过程,
也可以被看作一个数学对象,用于推导证明一些命题。在这篇介绍文章里,我也
打算从两个方面来讲,先讲它作为数学理论这方面的内容。

\section{数学理论}
让我们先从数学开始说起。函数是数学中一个非常基本的概念,我们很容易就能
写出一个计算平方和的函数如下:
\begin{equation}
  f(x,y) = x\times x + y\times y.
\end{equation}
上面这个平方和函数有个名字叫 $f$,有名字的好处是能方便的表示后续的计算,比如:
$$
f(3,4) = 3\times 3 + 4 \times 4 = 25.
$$但再想想,名字对一个函数来讲是必须的么?当然不是,下面这个映射也表示
了平方和函数:
\begin{equation}\label{eqn:mapfun}
  (x,y)\mapsto x\times x + y\times y
\end{equation}
把 (\ref{eqn:mapfun}) 整体当成函数的名字,同样可以表示计算平方和过程:
$$
((x,y)\mapsto x\times x + y\times y)(3,4)=3\times 3 + 4\times 4 = 25.
$$

我们把 (\ref{eqn:mapfun}) 称为匿名函数,它有两个参数。那么再问,单参数
函数和多参数函数的区分有必要吗?可以换个角度来看,我们把
(\ref{eqn:mapfun}) 改写成下面这个样子:
\begin{equation}\label{eqn:highfun}
  x\mapsto(y\mapsto x\times x + y \times y).
\end{equation}
这个映射是什么意思呢?$x$ 被映射成了 $y\mapsto x\times x + y \times y$,
后者是一个以 $y$ 为参数的函数。换句话说,(\ref{eqn:highfun}) 表示的是
把一个数映射成函数的函数,这就是所谓的``高阶函数''(High-Order
  Function)了。注意,(3) 现在是个单参数的函数,我们可以把它先应用于参
数 3,得到一个新的函数:
$$
(x\mapsto(y\mapsto x\times x + y \times y))(3)
=y\mapsto 3\times 3 + y \times y = y\mapsto 9 + y \times y.
$$
再把这个函数应用于参数 4:
$$
(y\mapsto 9 + y \times y)(4)=9+4\times 4=25.
$$这样就清楚了,(\ref{eqn:highfun}) 同样可以进行平方和的计算。用类
似的方法,可以把任意多参数函数都转换成单参数的高阶函数,这个转换又叫做
柯里化(Currying),这是以数学家 Haskell Brooks Curry 命名的。

匿名函数和柯里化,是 $\lambda$ 演算为简化函数概念而采取的方法,上面这
段说明可以看作是 $\lambda$ 演算的非正式介绍。事实上,大多数程序语言里
的所谓 Lambda 表达式,也就是这两个概念组合在一起罢了。下面正式介绍严格
的,形式化的 $\lambda$ 演算。

\subsection{形式化定义}\label{subsec:formal}
\begin{definition}{\bfseries{($\lambda$ 项)}}\label{def:lambdaterm}
  假设我们有一个无穷的字符串集合,里面的元素被称为\emph{变量}(和程序
    语言中变量概念不同,这里就是指字符串本身)。那么 \emph{$\lambda$ 项}定义如下:
  \begin{tightenum}
  \item 所有的变量都是 $\lambda$ 项(名为\emph{原子});
  \item 若 $M$ 和 $N$ 是 $\lambda$ 项,那么 $(M\,N)$ 也是 $\lambda$ 项
    (名为\emph{应用})
  \item 若 $M$ 是 $\lambda$ 项而 $\phi$ 是一个变量,那么 $(\lambda\phi.M)$
    也是 $\lambda$ 项(名为\emph{抽象})。
  \end{tightenum}
\end{definition}

上面定义的变量集合是可以任意指定的,比如在我写的解释器(见附录
  \ref{sec:interpreter})中,这个集合就是一般程序语言中可以用作变量名
的字符串\footnote{就是以英文字母开头,后面跟着有限多个字母或数字的字符
  串。}。在这篇文章里我们规定,变量集合是所有的小写英文字母及形如
$x',x'',x'''\dots$ 的字符串。

\begin{example}{\bfseries{(一些 $\lambda$ 项)}}
  下面这些都是 $\lambda$ 项:
  \begin{center}
    \begin{tabular*}{.8\textwidth}{@{\extracolsep{\fill} }ccc}
      $(\lambda x.(x\,y))$ & $(x(\lambda x.(\lambda x.x)))$ & $((((a\,b)\,c)\,d)\,e)$\\
      $(((\lambda x.(\lambda y.(y\,x)))\,a)\,b)$ & $((\lambda y.y)\,(\lambda x.(x\,y)))$ & $(\lambda x.(y\,z))$
    \end{tabular*}
  \end{center}
\end{example}

$\lambda$ 项是一种形式语言,换句话说,就是一类特殊形式的字符串罢了,没
有任何内在的意义,只是个``形式''。通常情况下,当讨论一个形式语言的时候,
我们需要用另一种元语言来指称形式语言里的元素。就如讨论自然数时,我们经
常说``对任意自然数 $n$'',这里的 $n$ 本身并不是自然数,用来指称自然数
罢了。我们也需要一些``$n$''来表示 $\lambda$ 项中的元素。因此,我们作如
下符号约定:

\begin{notation}
  本文中我们用大写英文字母表示任意 $\lambda$ 项,用除 $\lambda$ 以外的
  小写希腊字母如 $\phi$,$\psi$ 等表示任意 $\lambda$ 项中的变量。

  对于括号,则有如下的省略规定:
  \begin{tightenum}
  \item $\lambda$ 项中最外层的括号可以省略,如 $(\lambda x.x)$ 可以省
    略表示为 $\lambda x.x$;
  \item 左结合的应用型的 $\lambda$ 项,如 $(((M\,N)\,P)\,Q)$,括号可
    以省略,表示为 $M\,N\,P\,Q$;
  \item 抽象型的 $\lambda$ 项 $(\lambda \phi.M)$ 中,$M$ 最外层的括号可以省略,
    如 $\lambda x.(y\,z)$ 可以省略为 $\lambda x.y\,z$。
  \end{tightenum}
  也就是说,我们把省略形式视同定义 \ref{def:lambdaterm} 中的 $\lambda$ 项。
\end{notation}

\begin{example}{\bfseries{(省略表示)}}
  下面给出了一些省略表示的 $\lambda$ 项。
  \begin{center}
    \begin{tabular*}{.7\textwidth}{@{\extracolsep{\fill} }ll}
      省略表示 & 完整的 $\lambda$ 项\\
      $\lambda x.\lambda y.y\,x\,a\,b$ & $(\lambda x.(\lambda y.(((y\,x)\,a)\,b)))$\\
      $(\lambda x.\lambda y.y\,x)\,a\,b$ & $(((\lambda x.(\lambda y.(y\,x)))\,a)\,b)$\\
      $\lambda g.(\lambda x.g\,(x\,x))\,\lambda x.g\,(x\,x)$ & $(\lambda g.((\lambda x.(g\,(x\,x)))\,(\lambda x.(g\,(x\,x)))))$\\
      $\lambda x.\lambda y.a\,b\,\lambda z.z$ & $(\lambda x.(\lambda y.((a\,b)\,(\lambda z.z))))$\\
    \end{tabular*}
  \end{center}
\end{example}

\begin{example}
  用 $\lambda$ 演算解释器可以方便的查看一个 $\lambda$ 项的省略表示和完整
  表示。解释器用字符 \verb|\| 代表 $\lambda$。(更多关于这个解释器的用法和
    各个命令的意义,请参考附录 \ref{sec:interpreter}。)
  
\begin{verbcode}
    Lambda> :set +hold
    Lambda> (((\x.(\y.(y x))) a) b)
    (\x.\y.y x) a b
    Lambda> :set +fullform
    Lambda> (\x.\y.y x) a b
    (((\x.(\y.(y x))) a) b)
\end{verbcode}
\end{example}

\subsection{替换}

只定义了一个形式语言,那是没什么用处的。这一小节将形式化的定义什么是
$\lambda$ 项的替换操作。直观的来讲,就是把 $\lambda$ 项中不被
$\lambda\phi$ 约束的变量 $\phi$ 替换成另一个 $\lambda$ 项罢了。但这么
一个操作的精确定义却不是直接和易于理解的。我觉得这块的定义因严谨而优美,
但有的人看来可能就是琐碎而无聊了。没有兴趣的读者可以跳过这一小节。只要
记住 $\lambda$ 项的替换,是有精确严格定义的就行,反正具体计算可以让解
释器代劳嘛。

\begin{definition}{\bfseries{(语法全等)}}
  我们用恒等号``$\equiv$''表示两个 $\lambda$ 项完全相同。换句话说
  $$
  M\equiv N
  $$表示 $M$ 和 $N$ 有完全相同的结构,且对应位置上的变量也完全相同。这意
  味着若 $M\,N\equiv P\,Q$ 则 $M\equiv P$ 且 $N\equiv Q$,若
  $\lambda\phi.M\equiv\lambda\psi.P$ 则 $\phi\equiv\psi$ 且 $M\equiv P$。
\end{definition}

\begin{definition}{\bfseries{(自由变量)}}
  对一个 $\lambda$ 项 $P$,我们可以定义 $P$ 中\emph{自由变量}的集合
  $\text{FV}(P)$ 如下:
  \begin{tightenum}
  \item $\text{FV}(\phi) = \{\phi\}$
  \item $\text{FV}(\lambda\phi.M) = \text{FV}(M)\,\backslash\,\{\phi\}$\label{enum:fvforabs}
  \item $\text{FV}(M\,N) = \text{FV}(M)\cup\text{FV}(N)$
  \end{tightenum}
\end{definition}

从第 \ref{enum:fvforabs} 可以看出抽象 $\lambda\phi.M$ 中的变量 $\phi$
是要从 $M$ 中被排除出自由变量这个集合的。若 $M$ 中有 $\phi$,我们可以
说它是被\emph{约束}的。据此可以进一步定义\emph{约束变量}集合。值得注意
的是,对同一个 $\lambda$ 项来说,这两个集合的交集未必为空。

\begin{example}{\bfseries{(自由变量)}}
  \begin{center}
    \begin{tabular*}{.7\textwidth}{@{\extracolsep{\fill} }ll}
      $\lambda$ 项 $P$ & 自由变量集合 $\text{FV}(P)$\\
      $\lambda x.\lambda y.x\,y\,a\,b$ & $\{a,b\}$\\
      $a\,b\,c\,d$ & $\{a,b,c,d\}$\\
      $x\,y\,\lambda y.\lambda x.x$ & $\{x,y\}$
    \end{tabular*}
  \end{center}
\end{example}

上面最后一个例子里,最左边的 $x,y$ 是自由变量,而最右侧的 $x$ 则是约束
变量。若对 $\lambda$ 项 $P$ 有 $\text{FV}(P)=\emptyset$,则称 $P$ 是
\emph{封闭}的,这样的 $P$ 又称为\emph{组合子}。

\begin{definition}{\bfseries{(出现)}}
  对于 $\lambda$ 项 $P$ 和 $Q$,可以定义一个二元关系\emph{出现}。我们
  说 $P$ 出现在 $Q$ 中,是这样定义的:
  \begin{tightenum}
  \item $P$ 出现在 $P$ 中;
  \item 若 $P$ 出现在 $M$ 中或 $N$ 中,则 $P$ 出现在 $(M\,N)$ 中;
  \item 若 $P$ 出现在 $M$ 中或 $P\equiv\phi$,则 $P$ 出现在 $(\lambda\phi.M)$ 中。
  \end{tightenum}
\end{definition}

有了上面这些定义,我们终于可以定义什么叫 $\lambda$ 项的替换操作了:
\begin{definition}{\bfseries{(替换)}}\label{def:sub}
  对任意 $M,N,\phi$,定义 $[N/\phi]\,M$ 是把 $M$ 中出现的自由变量
  $\phi$ 替换成 $N$,并改变部分约束变量名称以避免冲突后得到的结果。具
  体精确定义是一个对 $M$ 的归纳定义:
  \begin{tightenum}
  \item $[N/\phi]\,\phi\equiv N$
  \item $[N/\phi]\,\alpha\equiv\alpha$\hfill 对所有满足
    $\alpha\not\equiv\phi$ 的原子 $\alpha$
  \item $[N/\phi]\,(P\,Q)\equiv([N/\phi]\,P\;[N/\phi]\,Q)$\label{enum:sub:app}
  \item $[N/\phi]\,(\lambda\phi.P)\equiv\lambda\phi.P$\label{enum:sub:samebind}
  \item $[N/\phi]\,(\lambda\psi.P)\equiv\lambda\psi.P$\hfill 若
    $\phi\not\in\text{FV}(P)$\label{enum:sub:bind}
  \item
    $[N/\phi]\,(\lambda\psi.P)\equiv\lambda\psi.[N/\phi]\,P$\hfill
    若 $\phi\in\text{FV}(P)$ 且 $\psi\not\in\text{FV}(N)$\label{enum:sub:1free}
  \item
    $[N/\phi]\,(\lambda\psi.P)\equiv\lambda\eta.[N/\phi][\eta/\psi]\,P$\hfill
    若 $\phi\in\text{FV}(P)$ 且 $\psi\in\text{FV}(N)$\label{enum:sub:2free}
  \end{tightenum}
  对其中从 \ref{enum:sub:bind} 到 \ref{enum:sub:2free} 的各条来说,
  $\phi\not\equiv\psi$;而对 \ref{enum:sub:2free},$\eta$ 是满足
  $\eta\not\in\text{FV}(N\,P)$ 的任意变量。
\end{definition}

下面解释一下这个看上去很长很恐怖的定义,其实只要带着``替换 $\lambda$
项中的自由变量 $\phi$ 为 $N$''这个直观去看,就不难理解。前两条无非是说,
要是一个原子刚好是要被替换的变量,那就替换,不然就不动。第
\ref{enum:sub:app} 条也好说,遇到应用了,那就替换各子项。第
\ref{enum:sub:samebind} 条,$P$ 中的 $\phi$ 是被约束的,不能替换所以结
果不变。同样的情况发生在第 \ref{enum:sub:bind} 条,$P$ 中没有自由变量
$\phi$,结果也不变。顺着这个思路往下想,$P$ 中要是有自由变量 $\phi$,
是不是就可以替换了呢?是的,这就是第 \ref{enum:sub:1free} 条的内容了。
不过这条还有个附加条件,$\psi\not\in\text{FV}(N)$,为什么呢?因为 $P$
是被 $\lambda\psi$ 约束着的,把 $N$ 替换进去的时候,要是
$\text{FV}(N)$ 里面有 $\psi$,那替换进去的 $N$ 中的 $\psi$ 就从自由变
量变成约束变量了。这多少有点让人不放心,所以我们先把这个条件加上,这样
替换不会把自由变量变成约束变量,这就是第 \ref{enum:sub:1free} 了。但要
是不如意事长八九,$\text{N}$ 中偏偏有 $\psi$ 怎么办呢?为了避免这个冲
突,干脆把原先约束的变量先换成绝对不会起冲突的吧,找一个
$\eta\not\in\text{FV}(N\,P)$ 换掉 $\psi$,然后再按第
\ref{enum:sub:1free} 条来办,这就是第 \ref{enum:sub:2free} 的内容了。
特别注意这一条中,不仅把 $P$ 中的 $\psi$ 换成了 $\eta$,连最前面的
$\lambda\psi$ 都被换成了 $\lambda\eta$。

上面的解释只是为了让读者理解一下定义 \ref{def:sub} 的合理性,其实还是
有一些没有解释清楚的疑问,比如为什么从自由变量变成约束变量就不好了?第
\ref{enum:sub:2free} 条中 $\eta$ 有无数种选择,这种不确定性会不会有什
么问题?这些问题的答案是:没什么为什么,定义如此。定义 \ref{def:sub}
是定义不是定理,它说了算,爱怎么定义怎么定义。

定义 \ref{def:sub} 第 \ref{enum:sub:2free} 条里,替换的不确定性暗示了
可以任意替换约束变量而不改变``意义''。这个可以类比的理解,例如定积分
$$
\int_a^bf(x)\text{d}x
$$里的 $x$ 是所谓的哑变量,换成别的如 $f(t)\text{d}t$ 不会改变积分式的
值。不过这只是我为了能直观理解而加上的一种解释。实际上 $\lambda$ 项是
形式系统,只是形式,一堆字符串罢了,哪里有什么``意义''。

可以说为了补救任意替换约束变量给人带来的不安,我们有了下面的定义:

\begin{definition}{\bfseries{($\alpha$ 变换和 $\alpha$ 等价)}}
  设 $\lambda\phi.M$ 出现在一个 $\lambda$ 项 $P$ 中,且设
  $\psi\not\in\text{FV}(M)$,那么把 $\lambda\phi.M$ 替换成
  $$
  \lambda\psi.[\psi/\phi]\,M
  $$的操作被称为 $P$ 的 \emph{$\alpha$ 变换}。当且仅当若 $P$ 经过有限
  步(包括零步)$\alpha$ 变换后,得到新的 $\lambda$ 项 $Q$,则我们可以
  称 $P$ 与 $Q$ 是 $\alpha$ 等价的,又写作
             $$ P\quad\equiv_\alpha\quad Q
             $$
\end{definition}

\begin{example}
  $$
  \lambda x.\lambda y.x(x\,y) \equiv_\alpha\lambda x.\lambda v.x(x\,v) \equiv_\alpha\lambda u.\lambda v.u(u\,v)
  $$
\end{example}

有了 $\alpha$ 等价,定义 \ref{def:sub} 就显得更加合理了,第
\ref{enum:sub:2free} 带来的疑问也得到解答了:不是担心任意替换不妥当么?
那么我们把 $\alpha$ 等价的 $\lambda$ 项都看成相同的,这不就``补救''了
由于不确定替换带来的问题嘛。事实上,后面的相关的定理也是这么做的。

强调一下,这个小节里定义的 $[N/\phi]\,M$ 并不是 $\lambda$ 项这个形式语
言里的东西,它只是一种记号,用来指称替换后的 $\lambda$ 项而已,切记切
记。而我各种解释说明中的意义和类比都只是为了方便直观的理解,并不是说
$\lambda$ 项有这些``意义''。它只是``名'',和``实''不相关。整个这篇文章
想表达的就是,某些``实'',完全可以用对 $\lambda$ 项这个``名''的各种操
作来体现。

\subsection{规约}

现在我们有了 $\lambda$ 项,有了替换的规则,那么什么时候需要进行替换呢?
这就是本小节要讨论的话题:规约。有了规约,整个关于 $\lambda$ 项的形式
系统才算完整,才可以说是 $\lambda$ 演算。

先介绍 $\beta$ 规约,$\beta$ 规约可以这么直观的理解:我们可以把
$(\lambda\phi.M)$ 看成是参数为 $\phi$,函数体为 $M$ 的一个函数;把
$(M\,N)$ 看成是函数 $M$ 作用到实际参数 $N$ 上\footnote{还是再强调一下,
  这只是直观的理解,并不是说两种 $\lambda$ 项一个是函数,一个是函数应
  用。$\lambda$ 项不是这些意义,只是字符串。}。平时我们要是定义了函数
$f(x)=x+5$,那么函数应用 $f(6)$ 就是把 $x+5$ 中的 $x$ 替换成 6,得到
$f(6)=6+5=11$。替换是函数应用的实质啊。

\begin{definition}{\bfseries{($\beta$ 规约)}}
  形如
  $$
  (\lambda\phi.M)\,N
  $$
  的 $\lambda$ 项被称为 \emph{$\beta$ 可约式},对应的项
  $$
  [N/\phi]\,M
  $$则称为 \emph{$\beta$ 缩减项}。当 $P$ 中含有 $(\lambda\phi.M)\,N$
  时,我们可以把 $P$ 中的 $(\lambda\phi.M)\,N$ 整体替换成
  $[N/\phi]\,M$,用 $R$ 指称替换后的得到的项,那么我们说 $P$
  \emph{$\beta$ 缩减}为 $R$,写做:
  $$
  P\;\triangleright_{1\beta}\;R
  $$当 $P$ 经过有限步(包括零步)的 $\beta$ 缩减后得到 $Q$,则称 $P$
  \emph{$\beta$ 规约}到 $Q$,写做:
  $$
  P\;\triangleright_\beta\;Q
  $$
\end{definition}

\newpage
\appendix
%% \section{\texorpdfstring{$\lambda$ 演算解释器简要使用说明}{$Lambda 演算解释器简要使用说明}}
\section{解释器简要使用说明}\label{sec:interpreter}

在学习了解 $\lambda$ 演算的过程中,为了加深自己对 $\lambda$ 演算的感性
认识,我实现了一个简单的解释器。这个解释器只实现了两个简单的功能:对无
类型纯 $\lambda$ 演算进行 $\beta$ 规约,允许命名绑定组合子。有了这两个
简单的功能,就可以把 $\lambda$ 演算当作一种程序设计语言,真实不虚的
``运行''这篇文章里介绍的各种 $\lambda$ 项,看到归约后的结果。用户还可
以对解释器做一些设置,决定输出形式,是否逐步给出规约过程等等。

解释器可以从这个地方下载:\downloadurl。这个解释器中用字符 \verb|\|
代表 $\lambda$,其余语法和 \ref{subsec:formal} 中定义的一样。下面介绍
一下用户可以设定的参数。

默认设置下,解释器会自动对用户输入的 $\lambda$ 项不断进行 $\beta$ 规约,
直到得到一个范式(如果存在的话)。可用命令 \verb|:set +hold| 让解释器
进入``原样输出''模式,命令 \verb|:set -hold| 则恢复立即执行的模式。

\begin{verbcode}
  Lambda> (\x.\y.y x) a b
  b a
  Lambda> :set +hold
  Lambda> (\x.\y.y x) a b
  (\x.\y.y x) a b
\end{verbcode}

解释器默认的输出样式采用了省略表示,去掉了不必要的括号,可以用命令
\verb|:set +fullform| 和 \verb|:set -fullform| 控制是否输出全部带括号
的结果。

\begin{verbcode}
  Lambda> (a b) c d e f g
  a b c d e f g
  Lambda> :set +fullform
  Lambda> a b c d e f g
  ((((((a b) c) d) e) f) g)
  Lambda> \x.\y.y x a b
  (\x.(\y.(((y x) a) b)))
\end{verbcode}

解释器默认情况下,只输出最终的规约结果,可以用命令 \verb|:set +trace|
和 \verb|:set -trace| 控制是否输出中间结果。

\begin{verbcode}
  Lambda> (\x.\y.y x) a b
  b a
  Lambda> :set +trace
  Lambda> (\x.\y.y x) a b
  ==> (\x.\y.y x) a b
  ==> (\y.y a) b
  ==> b a
\end{verbcode}

由于一个 $\lambda$ 项是否一定可以被 $\beta$ 规约规约到 $\beta$ 范式是
不可判定的,所以为了折衷解决这个问题,解释器设置了一个阈值,当规约步数
超过``$\lambda$ 项长度$\times$阈值'' 时,解释器就输出说疑似不可规约。
这个阈值可以用 \verb|:set steps 非负整数| 来修改,默认是 100。用户若是
觉得阈值及前述三个控制选项经多次修改之后太乱了,可以用命令
\verb|:reset settings| 来恢复默认设置。

\begin{verbcode}
  Lambda> (\x.\y.y y x) a b
  b b a
  Lambda> :set steps 0
  Lambda> (\x.\y.y y x) a b
  (\x.\y.y y x) a b seems can't be reduced!
  Lambda> :reset settings
  Lambda> (\x.\y.y y x) a b
  b b a
\end{verbcode}

解释器支持用户绑定名称到组合子,注意,只能是组合子,也就是没有自由变量
的 $\lambda$ 项。组合子中可以出现已经绑定的名称,但不允许递归定义。已
经绑定的名称不能再次绑定。
\begin{verbcode}
  Lambda> swap = \x.\y.y x
  Lambda> swap a b
  b a
  Lambda> foo = swap \x.x
  Lambda> foo \x.y
  y
  Lambda> id = id \x.x
  id can't be recursively defined!
  Lambda> foo = \x.x
  foo has been defined already!
\end{verbcode}

解释器已经默认绑定了一些这篇文章里出现的组合子,全部列表可见
\ref{sec:bindCombinator}。用户在添加了自己的绑定后,可以用
\verb|:reset state| 来恢复默认的绑定,清空自定义的绑定。也可以用命令
\verb|:clear state| 来清空全部的绑定。
\begin{verbcode}
  Lambda> plus
  \m.m \n.\f.\x.f (n f x)
  Lambda> plus four four
  \f.\x.f (f (f (f (f (f (f (f x)))))))
  Lambda> id = \x.x
  Lambda> id a
  a
  Lambda> :reset state
  Lambda> id a
  id a
  Lambda> :clear state
  Lambda> plus four four
  plus four four
  Lambda> :reset state
  Lambda> plus four four
  \f.\x.f (f (f (f (f (f (f (f x)))))))
\end{verbcode}

以上就是用户在 $\lambda$ 演算解释器里可以做的全部设置了。要退出解释器,
输入 \verb|:q| 命令即可。

\section{解释器默认绑定的组合子和名称}\label{sec:bindCombinator}
其实就是依次运行如下命令后,解释器内部状态中绑定的组合子。
\begin{verbcode}
  zero   = \f.\x.x
  succ   = \n.\f.\x.f (n f x)
  plus   = \m.\n.m succ n
  mult   = \m.\n.\f.m (n f)
  pow    = \b.\e.e b
  pred   = \n.\f.\x.n (\g.\h.h (g f)) (\u.x) (\u.u)
  sub    = \m.\n.n pred m
  one    = succ zero
  two    = succ one
  three  = succ two
  four   = succ three
  true   = \x.\y.x
  false  = \x.\y.y
  and    = \p.\q.p q p
  or     = \p.\q.p p q
  not    = \p.\a.\b.p b a
  if     = \p.\a.\b.p a b
  iszero = \n.n (\x.false) true
  leq    = \m.\n.iszero (sub m n)
  eq     = \m.\n. and (leq m n) (leq n m)
  Y      = \g.(\x.g (x x)) (\x.g (x x))
\end{verbcode}
\end{document}
